\documentclass[12 pt]{article}
\usepackage[utf8]{inputenc}
\usepackage{matlab-prettifier}
\usepackage[portuguese]{babel}
\usepackage{indentfirst}
\usepackage{graphicx}
\usepackage{float}
\usepackage{subcaption}
\usepackage[font=small,labelfont=bf]{caption}
\definecolor{mygreen}{RGB}{28,172,0} % color values Red, Green, Blue
\definecolor{myyellow}{rgb}{1.0, 1.0, 0.8}
\usepackage{mathtools}
\usepackage{multirow}
\usepackage{comment}
\usepackage{xcolor}
\usepackage{colortbl}
\usepackage[normalem]{ulem}               % to striketrhourhg text
\usepackage{amsmath}
\usepackage{amsfonts}
\newcommand\redout{\bgroup\markoverwith
{\textcolor{red}{\rule[0.5ex]{2pt}{0.8pt}}}\ULon}
\renewcommand{\lstlistingname}{Código}% Listing -> Algorithm
\renewcommand{\lstlistlistingname}{Lista de \lstlistingname s}% List of Listings -> List of Algorithms

\usepackage[top=3cm,left=2cm,bottom=2cm, right=2cm]{geometry}

% Configuração para destacar a sintaxe do Python
\lstset{ 
    language=Python,                     % A linguagem do código
    backgroundcolor=\color{myyellow}, % A cor do fundo 
    basicstyle=\ttfamily\footnotesize,   % O estilo do texto básico
    keywordstyle=\color{blue},           % Cor das palavras-chave
    stringstyle=\color{red},             % Cor das strings
    commentstyle=\color{mygreen},          % Cor dos comentários
    numbers=left,                        % Números das linhas à esquerda
    numberstyle=\tiny\color{gray},       % Estilo dos números das linhas
    stepnumber=1,                        % Número de linhas entre os números das linhas
    frame=single,                        % Moldura ao redor do código
    breaklines=true,                     % Quebra automática das linhas longas
    captionpos=t,                        % Posição da legenda
    showstringspaces=false               % Não mostra espaços em branco nas strings
    extendedchars=true,
    literate={º}{{${ }^{\underline{o}}$}}1 {á}{{\'a}}1 {à}{{\`a}}1 {ã}{{\~a}}1 {é}{{\'e}}1 {É}{{\'E}}1 {ê}{{\^e}}1 {ë}{{\"e}}1 {í}{{\'i}}1 {ç}{{\c{c}}}1 {Ç}{{\c{C}}}1 {õ}{{\~o}}1 {ó}{{\'o}}1 {ô}{{\^o}}1 {ú}{{\'u}}1 {â}{{\^a}}1 {~}{{$\sim$}}1
}


\title{%
\textbf{\huge Universidade Federal do Rio de Janeiro} \par
\textbf{\LARGE Instituto Alberto Luiz Coimbra de Pós-Graduação e Pesquisa de Engenharia} \par
\includegraphics[width=8cm]{COPPE UFRJ.png} \par
\textbf{Programa de Engenharia de Sistemas e Computação} \par
\large
CPS769 - Introdução à Inteligência Artificial e Aprendizagem Generativa \newline \par
\small
Prof. Dr. Edmundo de Souza e Silva (PESC/COPPE/UFRJ)\par 
Profa. Dra. Rosa M. Leão (PESC/COPPE/UFRJ)\par 
Participação Especial: Gaspare Bruno (Diretor Inovação, ANLIX) \par

\vspace{1\baselineskip}
\Large
\textbf{\textit{Lista de Exercícios 1a
}}
}

\author{Luiz Henrique Souza Caldas\\email: lhscaldas@cos.ufrj.br}

\date{\today}

\begin{document}
\maketitle

\section*{Questão 1}

Esse exemplo simples é para auxiliar a discussão do artigo “Serial Order A Parallel Distributed Processing Approach” que todos já devem ter lido. O objetivo é prever um padrão de figura, por exemplo um quadrado, usando uma Rede Neural Recorrente (RNN). Fornecemos o código em Python de um exemplo de geração do padrão 2-D de quadrados e treinamento de uma RNN para prever a sequência cíclica [0, 25, 0, 25], [0, 75, 0, 25], [0, 75, 0, 75], [0, 25, 0, 75], [0, 25, 0, 25].

\begin{enumerate}
    \item Entenda o código e explique qual a RNN que ele modela (faça o desenho). Explique a parte do código que define a RNN.
    
    \textbf{Resposta:} \par

    \item Treine a rede. Aprenda como fazer, e explique.
   
    \textbf{Resposta:} \par

    \item Faça a previsão de algumas trajetórias, quando o ponto inicial varia. O que você conclui?
   
    \textbf{Resposta:} \par

    \item Modifique a RNN usada e observe o que acontece.

    \textbf{Resposta:} \par

    \item Quais os pontos principais que você concluiu do artigo “Serial Order A Parallel Distributed Processing Approach”?
        
    \textbf{Resposta:} \par

\end{enumerate}

\lstinputlisting[language=Python,caption=código fornecido]{Lista_1a.py}

\end{document}